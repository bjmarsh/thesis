\chapter{The CMS Experiment}

In the previous chapter we summarized the theoretical underpinnings of modern
particle physics, and discussed possible extensions that point towards
future research. Now we turn our attention to the machines that make such
research possible. The Large Hadron Collider (LHC) is the largest
particle collider ever built, smashing protons together at record energies.
On it are located four major experiments, designed to record electrical
snapshots of the collisions and allow physicists to reconstruct the details
of each event. Sec.~\ref{sec:lhc} describes the design and operation of the LHC,
and Sec.~\ref{sec:cms_det} introduces the Compact Muon Solenoid (CMS) experiment,
one of the four at the LHC and the one used for the analysis in this dissertation.

\section{The Large Hadron Colllider}
\label{sec:lhc}

\section{The CMS detector}
\label{sec:cms_det}

\section{Computing and reconstruction pipeline}
