\chapter{Results and Interpretation}
\label{chap:results}

The background estimation techniques described in the previous chapters are used to make predictions
in each of the 282 signal regions bins. We then compare and fit to observed event counts, using a
maximum-likelihood fit that takes into account all statistical and systematic uncertainties.
Finally, the results of this fit are used to constrain a variety of beyond-the-standard-model physics
scenarios. Sec.~\ref{sec:prefit} presents the raw, pre-fit results, Sec.~\ref{sec:fits} briefly describes
the fitting and limit-setting procedures, and Secs.~\ref{sec:susy_interp} and \ref{sec:lq_interp} show 
the resulting limits placed on all considered signal models.

\section{Pre-fit results}
\label{sec:prefit}

Figures~\ref{fig:results_mono_vl} to \ref{fig:results_incl} show comparisons of the pre-fit background estimates
and observed yields in all signal regions. The hatched regions in the upper panels (and the solid gray bands
in the ratio panels) represent the total statistical and systematic uncertainty in the background prediction.
For the monojet region, the $x$ axis binning is $\pt^\mrm{jet1}$ (in\GeV), and for the $\Nj\geq2$ regions
the $x$ axis binning is \mttwo (in\GeV). Dashed lines separate the bins into categories of \Nj and \Nb.
Observed yields are statistically consistent with the estimated SM background.

\begin{figure}[htbp]
  \begin{center}
    \includegraphics[width=0.93\textwidth]{figs/results/prefit_monojet_ratio.pdf} \\
    \includegraphics[width=0.93\textwidth]{figs/results/prefit_HT250to450_ratio.pdf} \\
    \caption{Expected (pre-fit) and observed yields for the full dataset collected from
      2016--18, corresponding to an integrated luminosity of \Lint. On top are the monojet
      signal regions, separated into \Nb categories and with $\pt^\mrm{jet1}$ binning on the $x$ axis.
      On the bottom are the $\Nj\geq2$ signal regions with $250 \leq \Ht < 450\GeV$, separated into
      topological regions with the \mttwo binning on the $x$ axis.
            }
    \label{fig:results_mono_vl}
  \end{center}
\end{figure}

\begin{figure}[htbp]
  \begin{center}
    \includegraphics[width=0.93\textwidth]{figs/results/prefit_HT450to575_ratio.pdf} \\
    \includegraphics[width=0.93\textwidth]{figs/results/prefit_HT575to1200_ratio.pdf} \\
    \caption{Expected (pre-fit) and observed yields for the full dataset collected from
      2016--18, corresponding to an integrated luminosity of \Lint. The top and bottom figures
      contain the $\Nj\geq2$ signal regions for the $450 \leq \Ht < 575\GeV$ and $575 \leq \Ht < 1200\GeV$
      regions, respectively. \mttwo binning is shown on the $x$ axis, with the dashed lines separating
      bins by topological region.
            }
    \label{fig:results_l_m}
  \end{center}
\end{figure}

\begin{figure}[htbp]
  \begin{center}
    \includegraphics[width=0.93\textwidth]{figs/results/prefit_HT1200to1500_ratio.pdf} \\
    \includegraphics[width=0.93\textwidth]{figs/results/prefit_HT1500toInf_ratio.pdf} \\
    \caption{Expected (pre-fit) and observed yields for the full dataset collected from
      2016--18, corresponding to an integrated luminosity of \Lint. The top and bottom figures
      contain the $\Nj\geq2$ signal regions for the $1200 \leq \Ht < 1500\GeV$ and $\Ht\geq1500\GeV$
      regions, respectively. \mttwo binning is shown on the $x$ axis, with the dashed lines separating
      bins by topological region.
            }
    \label{fig:results_h_uh}
  \end{center}
\end{figure}

\begin{figure}[htbp]
  \begin{center}
    \includegraphics[width=0.93\textwidth]{figs/results/prefit_inclusive.pdf}
    \caption{Expected (pre-fit) and observed yields for the full dataset collected from
      2016--18, corresponding to an integrated luminosity of \Lint. Here the results are shown
      integrated over \mttwo, with each bin on the $x$ axis corresponding to an \Nj/\Nb topological
      region. Dashed lines separate the various \Ht regions. The two leftmost regions contain the monojet
      signal regions, where the $x$ axis binning is $\pt^\mrm{jet1}$.
            }
    \label{fig:results_incl}
  \end{center}
\end{figure}


\section{Maximum-likelihood fits and the \texorpdfstring{CL$_\text{S}$}{CLs} technique}
\label{sec:fits}

The results presented in the previous section are \textit{pre-fit}, meaning the shown backgrounds
and uncertainties are straight from the estimation methods, without attempting to fit to the data
in any way. For the purposes of evaluating the results and setting constraints on models of new physics,
we apply a fitting procedure to test the compatibility of the observed results with SM predictions.

The expected background and signal yields are functions of \textit{nuisance parameters}, which control
the potential variations from all of the experimental and theoretical uncertainties. In practice these
are modeled as log-normal constraints on the background and signal yields (except for statistical uncertainties
on the observed control region event counts, which are modeled as gamma functions). Denoting the vector
of all nuisance parameters as $\boldsymbol\theta$, and their joint probability distribution as $p(\boldsymbol\theta)$, we can write
the complete likelihood function as
\be\label{eq:likelihood}
\mathcal{L}(\mrm{data}|\mu,\boldsymbol\theta) = \prod_{j\in\mrm{bins}} \frac{[\mu s_j(\boldsymbol\theta)+b_j(\boldsymbol\theta)]^{n_j}}{n_j!}
e^{-[\mu s_j(\boldsymbol\theta)+b_j(\boldsymbol\theta)]} p(\boldsymbol\theta),
\ee
where $s_j(\boldsymbol\theta)$ and $b_j(\boldsymbol\theta)$ are the predicted signal and background yields 
in bin $j$ (and which are functions of the nuisance parameters), $n_j$ is the observed event count in bin 
$j$, and $\mu$ is the signal strength parameter that we seek to set a constraint on. The likelihood is 
simply a product of Poisson probability terms with expectation $\mu s_j+b_j$ and observation $n_j$, and the 
PDF of the nuisance parameters, which is itself a product of log-normal and gamma functions.

To evaluate how well the estimated background fits the observed data, we could simply maximize this 
likelihood under the background-only hypothesis (i.e., set $\mu=0$ in Eq.~\ref{eq:likelihood}.
However, we would like to interpret this in the context of particular signal models, so we need a method
to evaluate the significance of the signal+backgroud (S+B) assumption (the \textit{alternative hypothesis})
with respect to the background-only (B-only) assumption (the \textit{null hypothesis}).

To do this, we make use of the CL$_\mrm{S}$ technique~\cite{Read:CLs}.
First, a test statistic is defined as
\be
q_\mu = -2\log\frac{\mathcal{L}(\mu, \hat{\boldsymbol\theta}_\mu)}{\mathcal{L}(0, \hat{\boldsymbol\theta}_0)},
\ee
where the likelihood $\mathcal{L}$ is as defined in Eq.~\ref{eq:likelihood}, $\hat{\boldsymbol\theta}_0$
is the value of $\boldsymbol\theta$ that maximizes $\mathcal{L}$ for $\mu=0$, and $\hat{\boldsymbol\theta}_\mu$ is
the value of $\boldsymbol\theta$ that maximizes $\mathcal{L}$ for the given $\mu$. This is known as a
\textit{profile likelihood}, as we have profiled away the dependence on the nuisance parameters $\boldsymbol\theta$
by maximizing over them. Note that the denominator is independent of $\mu$, and is only there as a normalization.

We can construct distributions of $q_\mu$ for a given $\mu$ by generating random data, either under the B-only
or $\mu$S+B hypotheses. The values of $q_\mu$ under the $\mu$S+B hypothesis tend to be lower on average than those 
under the B-only hypothesis, the numerator will be larger than the denominator. The distributions converge as
$\mu\to0$.

Using the actual observed data counts, we can then calculate the observed $q_\mu^\text{obs}$ for arbitrary $\mu$.
We then define
\be
\text{CL}_\text{S}(\mu) \equiv \frac{\text{CL}_\text{S+B}}{\text{CL}_\text{B}} \equiv
\frac{p(q_\mu\geq q_\mu^\text{obs}\;|\;\mu\text{S+B})}{p(q_\mu\geq q_\mu^\text{obs}\;|\;\text{B})}
\ee

Note that $\text{CL}_\text{S}(\mu)$ is exactly equal to 1 at $\mu=0$, and is a strictly decreasing function of $\mu$.
One can think of $\text{CL}_\text{S}$ as the relative likelihood that the observed counts would look as ``background-like''
as they do under the $\mu$S+B hypothesis versus the B-only hypothesis. When CL$_\text{S}$ is very small,
we can be pretty confident that the $\mu$S+B assumption is \textit{not} supported by the data. Thus, we define
an upper limit on $\mu$ at the $1-\alpha$ confidence level as the $\mu_{1-\alpha}$ that satisfies
\be
\text{CL}_\text{S}(\mu_{1-\alpha}) = \alpha
\ee

This process is illustrated in Fig.~\ref{fig:results_cls} for a toy experiment with 3 bins. We predict event counts of 
10, 5, and 2, with an uncorrelated 20\% uncertainty on each. Nominal expected signal with $\mu=1$ is
2 events in each bin. We observe event counts of 9, 5, and 1. The top left panel shows the expected and observed
yields in each bin. The top right panel shows distributions of $q_{\mu=1}$ under the B and S+B hypotheses, the
observed value of $q_{\mu=1}^\text{obs}=3.73$, and shaded areas representing CL$_\text{S+B}$ and CL$_\text{B}$
The bottom panel shows CL$_\text{S}(\mu)$ as a function of $\mu$, and the 95\% confidence level upper limit
on $\mu$ of 1.19, computed by finding the $\mu$ value at which CL$_\text{S}(\mu)$ crosses below 0.05.

\begin{figure}[htbp]
  \begin{center}
    \includegraphics[width=0.48\textwidth]{figs/results/cls.pdf}
    \includegraphics[width=0.48\textwidth]{figs/results/qmu_dist.pdf} \\
    \includegraphics[width=0.48\textwidth]{figs/results/cls.pdf}
    \caption{blah
            }
    \label{fig:results_cls}
  \end{center}
\end{figure}

The $\text{CL}_\text{S}$ method outlined above is used to set limits on all of the signal models in the following
sections. Due to the very large number of bins in the \mttwo analysis, it is computationally infeasible
to do compute distributions of the test statistic explicitly. Hence, we use an \textit{asymptotic approximation} to
derive the distributions necessary to calculate $\text{CL}_\text{S}$, as described in~\cite{Cowan:asymptotic}.


\section{Supersymmetry interpretations}
\label{sec:susy_interp}

\section{Leptoquark interpretations}
\label{sec:lq_interp}
