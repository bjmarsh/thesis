\chapter{Event Selection and Triggering}

The general strategy of the \mttwo analysis is to apply a baseline selection
motivated by the available triggers and by the desire to reduce QCD multijet
background to manageable levels, and then to categorize the selected events
into bins of differing levels of hadronic activty, number of b-tagged jets, 
and missing transverse energy. The ``namesake'' variable \mttwo, used to reduce
QCD background, was described in Sec.~\ref{sec:mt2_variable}, but a number of other
variables are used to constrain backgrounds and categorize events.
Sec.~\ref{sec:objvardefs} defines the
relevant physics objects (jets, leptons, etc.) and kinematic variables, Sec.~\ref{sec:triggers} describes
the triggers used in the analysis, Sec.~\ref{sec:baselinesel} outlines the
baseline selection, and Sec.~\ref{sec:srdefs} gives the precise definitions
of the signal regions.

\section{Object and variable definitions}
\label{sec:objvardefs}

In CMS, individual particles are identified by combining information from the tracker, calorimeters, and muon system
using the ``particle flow'' algorithm, described in Sec.~\ref{sec:cms_det}. This particle-level information is
then used to cluster jets, reconstruct vertices, and compute the missing transverse energy. Here we list the various
physics objects used in this analysis and the selections applied on them.

\subsection{Vertices}
``Vertex finding'' is a process of finding points in space from which groups of reconstructed particle tracks that, loosely,
come from the same ``interaction'', emanate. There is generally a single ``primary vertex'', where the hard interaction
took place, pileup vertices from pileup interactions, and, potentially, secondary vertices from longer-lived decaying particles
such as b hadrons. Algorithms for reconstructing vertices are described in \cite{TRK_vertexing}. For this analysis,
we consider a reconstructed vertex as good if it satisfies:
\begin{itemize}
\item not ``fake'' - if no vertices are reconstructed from tracks, a default vertex based on the beam-spot 
(luminous region produced by proton beam collision) is used, and is labeled as ``fake''.
\item $N_\mrm{dof}>4$ - $N_\mrm{dof}$ is the number of degrees of freedom in the fit of the position
of the vertex, essentially the number of tracks consistent with originating from the vertex.
\item $|z|<25$ cm - the longitudinal distance from the beam-spot.
\item $\rho<2$ cm - the distance from the beam axis.
\end{itemize}

When more than one good reconstructed vertex is found in the event, the reconstructed vertex
with the largest value of summed physics-object $\pt^2$ is taken to be the primary interaction
vertex.

\subsection{Jets}

Jets are clustered using the anti-$k_\mrm{T}$ algorithm with distance parameter $R=0.4$.
Charged hadrons from pileup interactions are identified and removed based on the ``charged hadron
subtraction'' algorithm \cite{JME_pileup_removal_algo}.
Jet energies are corrected for pileup contamination and detector response with CMS-derived
era-dependent jet energy corrections.
We select jets that satisfy $\pt>30~\GeV$ and $|\eta|<2.4$, and pass PF jet loose ID (2016)
or tight ID (2017+18). For events with only one jet, we require tighter ID requirements to reject noisy jets. 
All jet ID cuts are summarized in Table~\ref{tab:jet_id}.

\begin{table}[h]
\caption{Jet ID definitions}
\label{tab:jet_id}
\centering
\begin{tabular}{l|c|c|c}
\hline
 & 2016 & 2017-18 & Monojet (all years) \\ \hline
Neutral hadron fraction & $<0.99$ & $<0.90$ & $<0.80$ \\
Neutral EM fraction & $<0.99$ & $<0.90$ & $<0.70$ \\
Number of constituents & $>1$ & $>1$ & $>1$ \\
Charged hadron fraction & $>0$ & $>0$ & $>0.05$ \\
Charged multiplicity & $>0$ & $>0$ & $>0$ \\
Charged EM fraction & $<0.99$ & -- & $<0.99$ \\
\hline
\end{tabular}
\end{table}

We define \Nj as the number of jets passing the above selections, and \Ht as 
the scalar sum of \pt values for all such jets.

Jets originating from b quarks are tagged using the DeepCSV algorithm \cite{BTV_btagging}, at the medium working point. 
For the purposes of counting the number of b-tags (\Nb), we loosen the \pt threshold for b-tagged jets to 20~\GeV.
This helps to add sensitivity to compressed-spectrum signals with jets from b quarks.

\subsection{\ptmiss}
We use type 1-corrected PFMET, defined in Sec.~\ref{sec:jetmet}, using the same jet energy corrections
as applied to the jets. We additionally define \vMht as
\be
\vMht = -\sum_\mrm{jets} \vec{p}_{\mrm{T},i},
\ee
where the sum is taken over all jets passing the above requirements. The difference with respect to \vMet is that
this excludes forward or low-\pt jets and unclustered energy.

\subsection{\ptmiss filters}
In addition to real missing energy due to invisible particles, events may have some amount of ``fake \ptmiss'' due to either
detector effects or external sources (e.g. cosmic rays or beam-halo particles). We have discussed fake \ptmiss from 
standard jet mis-measurement due to stochaistic smearing in the calorimeters, but more pathological effects are also possible,
such as noisy calorimeter cells or bad track reconstructions. To eliminate as best as possible events containing such sources
of fake \ptmiss, the JetMET group at CMS recommends a set of ``\ptmiss filters'' that use features of the reconstructed
event to identify certain classes of bad events. We apply all standard recommended filters, listed here:
\begin{itemize}\setlength\itemsep{-1mm}
\item primary vertex filter
\item CSC super-tight beam halo 2016 filter (despite name, used in all 3 years)
\item HBHE noise filter
\item HBHE iso noise filter
\item EE badSC noise filter
\item ECAL dead cell trigger primitive filter
\item bad muon filter
\item ECAL bad calibration filter (2017+18 only)
\end{itemize}

There are also a few custom \ptmiss filters developed by this analysis or the CMS SUSY group that are applied to protect 
against other observed sources of fake \ptmiss.
First, we reject any event containing a jet with $\pt>30~\GeV$ and $|\eta|<4.7$ which fails the PF jet loose/tight ID as described above.
Since this jet would not enter the collection used to compute \mttwo, the pseudojets would likely be imbalanced and the resulting
\mttwo biased.
This is not applied to fast simulation MC samples since the input variables are not correctly modeled, and \mttwo
is expected to already be high anyway.

Next, we require that the ratio of PFMET over caloMET (\ptmiss computed only with calorimeter deposits) is less than 5.
This was found to remove events with a bad high-\pt muon track inside a jet, which are not removed by either the lepton
vetoes or the bad muon track filter. 
Also to reduce the effect of mis-measured muons, we veto events that contain a jet with $\pt>200~\GeV$, a muon fraction
larger than 50\%, and satisfy $|\Delta\phi(\mrm{jet},\ptmiss)| > \pi-0.4$.

Finally, to remove certain known pathological events in fast simulation MC, we remove events in such MC containing
a jet satisfying $\pt>20~\GeV$, $|\eta|<2.5$, charged hadron fraction $<0.1$, and no matching generator-level jet
within $\Delta R<0.3$.

\subsection{Electrons}

While the analysis signal regions are purely hadronic, it is still necessary to define lepton candidates, 
both in order to define a lepton veto and to select events for the leptonic control regions.
Electron candidates are required to satisfy $\pt>10~\GeV$ and $|\eta|<2.4$. Good electrons are identified
using cut-based ID working points developed by the EGamma group at CMS: the ``veto'' working point is used for
the signal region veto and the single-lepton control region, and the ``loose'' working point is used for
the dileptonic \zll control region. The cuts are summarized in Table~\ref{tab:electron_id}, and definitions
for the various variables are listed here.
\begin{itemize}\setlength\itemsep{-1mm}
\item $\sigma_{i\eta i\eta}$ - a variable describing the width of the shower in the ECAL; computed using the 
reconstructed hits in a 5x5 seed cluster
\item $|\Delta\eta_\mrm{Seed}|$ - difference in $\eta$ between a ECAL cluster position and track direction at vertex extrapolated to ECAL
\item $|\Delta\phi_{In}|$ - difference in $\phi$ between a ECAL cluster position and track direction at vertex extrapolated to ECAL
\item $H/E$ - ratio of energy in HCAL behind ECAL cluster to the energy in the ECAL cluster
\item $|1/E-1/p|$ - tests consistency of ECAL cluster energy and track momentum
\item $|d0|$ - transverse impact parameter of the track with respect to the primary vertex
\item $|dz|$ - longitudinal impact parameter of the track with respect to the primary vertex
\item conversion veto - reject electron candidates that look like photon conversions to $e^+e^-$ pairs
\end{itemize}

\begin{table}[htbp]
\caption{Cut-based electron ID for the Veto and Loose ID working points, for electrons in the barrel (endcap).}
\label{tab:electron_id}
\scriptsize
\centering
\resizebox{\textwidth}{!}{%
\begin{tabular}{c|c|c}
\hline
&Veto ID & Loose ID \\
\hline
\hline
$\sigma_{i\eta i\eta}$ & \multirow{2}{*}{$ < 0.0126$ ($0.0457$)} & \multirow{2}{*}{$ < 0.0112$ ($0.0425$)}\\
(RecHits in 5x5 seed cluster)& &\\ 
\hline
$|\Delta\eta_\mrm{Seed}|$& $ < 0.00463$ ($0.00814$)& $ < 0.00377$ ($0.00674$) \\ 
\hline
$|\Delta\phi_{In}|$ &$ < 0.148$ ($0.19$) & $ < 0.0884$ ($0.169$) \\ 
\hline
\multirow{2}{*}{$H/E$} &  $ < 0.05+1.16/E_{\mathrm{SC}}+0.0324~\rho/E_{\mathrm{SC}}$ &$ < 0.05+1.16/E_{\mathrm{SC}}+0.0324~\rho/E_{\mathrm{SC}}$ \\
&($ < 0.05+2.54/E_{\mathrm{SC}}+0.183~\rho/E_{\mathrm{SC}}$)  & ($0.0441+2.54/E_{\mathrm{SC}}+0.183~\rho/E_{\mathrm{SC}}$)\\ 
\hline
$|\frac{1}{E} - \frac{1}{p}|$  &$ < 0.209$ ($0.132$) &$ < 0.193$ ($0.169$)\\ 
\hline
$|d0|$ (w.r.t. primary vertex) &$ < 0.2$ ($0.2$)~cm & $ < 0.2$ ($0.2$)~cm\\ 
\hline
$|dz| $(w.r.t. primary vertex)  &$ < 0.5$ ($0.5$)~cm  &$ < 0.5$ ($0.5$)~cm \\ 
\hline
\# of expected missing inner hits &$ \leq 2$ (3)&$ \leq 1$ (1)\\ 
\hline
conversion veto& yes & yes \\
\hline
\end{tabular}
}
\end{table}

In addition to the ID described above, electrons are required to be isolated. This is defined using relative
mini-PF isolation, as miniPFIso$/\pt<0.1$. ``PF isolation'' is just the sum of the \pt values of all particle flow candidates
within a cone around the electron candidate, and the ``mini'' part means that this cone size gets smaller
with higher electron \pt. Precisely, the cone size used is
\begin{equation}
\label{eqn:miniiso}
 \Delta R =
  \begin{cases}
   0.2          & \text{if } \pt < 50\GeV \\
   10\GeV/\pt   & \text{if } 50 < \pt < 200\GeV \\
   0.05         & \text{if } \pt > 200\GeV
  \end{cases}
\end{equation}

A correction to the isolation to account for pileup contamination is applied, based on the event-level energy
density and the effective area of the electron cone.

\subsection{Muons}

Muon candidates are required to pass $\pt>10~\GeV$ and $|\eta|<2.4$, and a loose ID selection defined as:
\begin{itemize}\setlength\itemsep{-1mm}
\item matched to a particle flow muon
\item either a global muon (tracker+muon system) or a tracker-only muon
\item $|d0|<0.2$~cm (transverse impact parameter with respect to the primary vertex)
\item $|dz|<0.5$~cm (longitudinal impact parameter with respect to the primary vertex)
\end{itemize}

We require the muons to be isolated using the same relative mini PF isolation as used for the electrons,
this time requiring miniPFIso$/\pt<0.2$.

\subsection{Isolated tracks}

In addition to vetoing events with the reconstructed leptons described above, we further add a veto
for events with ``isolated tracks'' that weren't reconstructed as leptons, either because
they are charged hadrons or they failed some criteria to be promoted to a full reconstructed lepton.
This allows better rejection of backgrounds with hadronically decaying $\tau$ leptons (these frequently
produce isolated pions) or with isolated leptons that weren't caught be the lepton veto, without
appreciably affecting signal efficiency.
We select charged particle flow candidates with different requirements depending on the type of candidate.

For particle flow electrons and muons, we require them to pass $\pt>5\GeV$, $|\eta|<2.4$, $|dz|<0.1$~cm,
$|dxy|<0.2$~cm, and a track isolation cut of iso$/\pt<0.2$. The track isolation is computed as the sum
of all charged hadron particle flow candidates with in a cone of $\Delta R<0.3$, and that satisfy 
$|dz|<0.1$~cm with respect to the primary vertex. For lepton counting, particle flow leptons
within $\Delta R<0.01$ of selected reconstructed leptons are removed.

Charged particle flow hadrons are required to pass $\pt>10\GeV$, $|\eta|<2.4$, $|dz|<0.1$~cm,
$|dxy|<0.2$~cm, and a track isolation cut of iso$/\pt<0.1$, computed in the same way as above.

\subsection{\dphimet}
The variable \dphimet (referred to as \dpmin in the following) is defined as the minimum $\Delta\phi$ between
\vMet and any of the four highest \pt jets in the event. For this variable only, we consider jets with
$\pt>30~\GeV$ and $|\eta|<4.7$.


\section{Triggers}
\label{sec:triggers}

\section{Baseline selections}
\label{sec:baselinesel}

\section{Signal region definitions}
\label{sec:srdefs}
