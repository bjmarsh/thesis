\begin{abstract}
\addcontentsline{toc}{chapter}{Abstract}

A search for phenomena beyond the Standard Model (BSM) is performed using events with 
hadronic jets and significant transverse momentum imbalance. The results are based on 
a sample of proton-proton collisions at a center-of-mass energy of $13\TeV$, collected 
by the CMS experiment at the Large Hadron Collider in 2016--2018 and corresponding to an integrated luminosity 
of \Lint. The search is based on signal regions defined by the hadronic 
energy in the event, the jet multiplicity, the number of b-tagged jets, 
and the value of the kinematic variable \mttwo for events with at least 
two jets. For events with exactly one jet, the transverse momentum of the jet is used 
instead. No significant excess event yield is observed above the predicted Standard Model background. This is used 
to constrain a range of BSM models that predict the following: the pair production of gluinos 
and squarks in the context of supersymmetry models conserving $R$-parity; the resonant production of a 
colored scalar state decaying to a massive Dirac fermion and a quark; and the pair production 
of scalar and vector leptoquarks each decaying to a neutrino and a top, bottom, or 
light-flavor quark. In most of the cases, the results obtained are the most stringent 
constraints to date. The analysis is published in the European Physical Journal C vol. 80, \#3.

Additionally, the first search at a hadron collider for elementary particles with charges much smaller
than the electron charge is presented. 
These results are based on a sample of proton-proton collisions at a center-of-mass energy of $13\TeV$ 
provided by the LHC in 2018, corresponding to an integrated luminosity of 37.5 fb$^{-1}$.
A prototype scintillator-based detector is deployed near the CMS interaction point to conduct a search 
sensitive to particles with charges ${\leq}0.3e$. 
The existence of new particles with masses between 20 and 4700\MeV is excluded at 95\% confidence level 
for charges varying between 0.006$e$ and 0.3$e$, depending on their mass. 
New sensitivity is achieved for masses larger than 700\MeV.

%\abstractsignature
\end{abstract}


