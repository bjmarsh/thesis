\chapter{Bonus: A Search for Milli-Charged Particles at the LHC}
{\small

\section{Motivation of a search for milli-charged particles}

One of the central mysteries of modern particle physics is the question of what
makes up dark matter. It must consist of massive particles that interact
at most very weakly with the SM, and no viable candidate exists among the
currently known particles. Moreover, a decade's worth of data from the LHC
has provided no evidence of any new particles that might provide an explanation.

We thus ask the question, what types of signatures might hypothetical dark matter
particles produce that would escape detection at present experiments?
One method of explaining dark matter is to add a new ``dark sector'' of
particles beyond the SM that couples only weakly to the SM. As an example,
we can add a ``dark photon'' $A'_\mu$ and a ``dark fermion'' $\psi'$ charged
under the new gauge field with charge $e'$. Allowing for kinetic
mixing between $A'_\mu$ and the SM weak hypercharge field $B_\mu$, the Lagrangian
for this new dark sector can be written
\be\label{eq:mcp_lagr}
\begin{split}
\mathcal{L}_\text{dark-sector} = &-\frac{1}{4}A'_{\mu\nu}A'^{\mu\nu} \\
&+i\bar{\psi}'(\gamma^\mu\partial_\mu + ie'\gamma^\mu A'_\mu + iM_\text{mCP})\psi' \\
&-\frac{\kappa}{2}A'_{\mu\nu}B^{\mu\nu},
\end{split}
\ee
where the first line is the kinetic term for a massless dark photon,
the second line contains the kinetic terms for a dark fermion with
mass $M_\mrm{mCP}$ as well as the interaction term with $A'_\mu$, and
the third line contains the mixing term between $A'_\mu$ and $B_\mu$, with
mixing strength parameter $\kappa$.

The mixing term can be eliminated by redefining $A'_{\mu\nu}\to A'_{\mu\nu}+\kappa B_{\mu\nu}$,
resulting in an interaction term $\kappa e'\bar{\psi}'\gamma^\mu B_\mu\psi$ between $\psi'$
and $B_\mu$. Rewriting $B_\mu$ in terms of the physical photon and $Z$ boson fields
as $B_\mu=\cos\theta_w A_\mu - \sin\theta_w Z_\mu$, we find that the new dark fermion
couples to the SM photon with electric charge $\kappa e'\cos\theta_w$, and couples to the 
SM $Z$ with charge $\kappa e'\sin\theta_w$. The mixing strength $\kappa$ must be small
(otherwise the new dark sector would have been observed already), so we must have
$\varepsilon\equiv \kappa e'\cos\theta_w \ll 1$, and we call $\psi'$ a
``milli-charged particle'' (mCP; note that the name is a bit of a misnomer because $\varepsilon$
does not have to be exactly $\mathcal{O}10^{-3}$))

\begin{figure}[t]
  \begin{center}
    \includegraphics[width=0.50\textwidth]{figs/milliq/search_status.png}
    \caption{Existing exclusion limits for milli-charged partcles, coming
      from searches via colliders, solar effects, astronomical observations,
      and cosmological bounds. milliQan targets the unexcluded phase space
      with $\varepsilon\geq10^{-3}$ and $10^{-1} < m_\text{mCP} < 10^2\GeV$.
      (Image from~\cite{Vinyoles:mcp})
            }
    \label{fig:mcp_search_status}
  \end{center}
\end{figure}

Such milli-charged particles have been searched for via a variety of methods,
either directly through collider experiments or indirectly through
solar effects, astronomical observations, or cosmological bounds.
A summary of the present exclusion space in the mCP mass--charge plane
is shown in Fig.~\ref{fig:mcp_search_status}, taken from~\cite{Vinyoles:mcp}.

There is a gap in the excluded phase space for $\varepsilon>10^{-3}$ at
the mass scales relevant at the LHC, roughly $10^{-1} < m_\text{mCP} < 10^2\GeV$.
mCPs at such masses and charges would be produced frequently at the LHC, but
present experiments would not be able to detect them; direct sensitvity is lost
for $\varepsilon$ below a few times $10^{-1}$, and a low cross section
precludes missing energy searches. Therefore, a dedicated experiment is necessary
to search for mCPs at the LHC.


\section{Overview of the milliQan detector}
The milliQan experiment, designed to search for mCPs using collisions at LHC P5, was proposed in 
2015~\cite{Haas:mcp,mq:loi}. It is located in a drainage gallery, elevated 45$^\circ$ above and 33
m from the CMS experiment, with 17 m of rock in between that naturally
suppresses beam-based backgrounds. The proposed design consists of three stacked layers of plastic 
scintillator arrays, with each scintillating bar coupled to a photomultiplier tube (PMT).
The arrays are pointed at the interaction point (IP), such that a particle originating
from a $pp$ collision would pass through all three layers in a straight line.
The bars are sensitive enough to detect individual photoelectrons produced by
throughgoing mCPs, and requiring a simultaneous hit in all three layers drastically
reduces background, which mostly consists of random overlap of pulses from
PMT dark rate, environmental radiation, cosmic rays, and afterpulsing.

The full milliQan design is anticipated to consist of three ``layers'' of $20\times20$ scintillator arrays,
each around 1 m$^2$ in total. In 2017--18, a smaller scale \textit{demonstrator} was installed
to study backgrounds and provide a proof-of-concept for the full-scale detector.
This demonstrator consists of three $2\times3$ scintillator bar arrays, roughly 1--2\%
of the planned full detector.

The 18 scintillator bars each measure 5 cm $\times$ 5 cm $\times$ 80 cm, and are
wrapped in layers of reflective and light-blocking materials to ensure optimal
efficiency. 3D-printed plastic casings couple the bars to individual PMTs
(two Hamamatsu R7725s, four Electron Tube 9814Bs, and 12 Hamamatsu R878s).

\begin{figure}[t]
  \begin{center}
    \includegraphics[width=0.440\textwidth]{figs/milliq/cavern_loc.png}
    \includegraphics[width=0.352\textwidth]{figs/milliq/demonstrator_illustration.png}
    \caption{(left) Illustration of the location of the drainage gallery with 
      respect to the CMS cavern. CMS is located in the large dome on the left; 
      milliQan is elevated 43.1$^\circ$ above this, and 33 m away, with 17 m of
      rock in between. (right) 3D illustration of the demonstrator detector.
      The slabs are in yellow, the panels in translucent green, and the bars can be seen through
      the panels. The gray blocks in the layer gaps are lead bricks to block radiation.
            }
    \label{fig:demonstrator}
  \end{center}
\end{figure}

In addition to the scintillator bars, there are four 20 cm $\times$ 30 cm $\times$ 2.5 cm
scintillator ``slabs'' placed at the front and rear of the detector
as well as in between the layers, in order to tag/veto beam-based and cosmic
particles. Additionally, each layer has three 18 cm $\times$ 102 cm $\times$ 0.7 cm scintillator
``panels'' covering the top and sides, in order to tag/veto cosmic particles and environmental radiation.
Each of the slabs and panels is read out by a Hamamatsu R878 PMT.

There are 5 cm-thick lead bricks placed between each layer to reduce correlated pulses from radiation.
All of the scintillators and lead bricks are mounted on a custom-designed aluminum support structure
that can rotate the entire stack in multiple directions to facilitate alignment with the IP.
A CERN engineering team performed such an alignment, so that the detector points
to the IP to within a tolerance of just 1 cm over the 33 m distance. The location of the drainage gallery
and an illustration of the demonstrator are shown in Fig.~\ref{fig:demonstrator}.

The 31 scintillator channels are read out by a pair of CAEN V1743 digitizers, which sample at 1.6 GHz
and record a 640 ns waveform for each channel upon triggering. The trigger can be configured
to fire on single, double, or triple coincidence of peaks at an arbitrary threshold.
For nominal data taking, the trigger is set to fire on a triple coincidence of pulses.

\section{Bench tests for PMT calibration}

\section{Generation of signal Monte Carlo}
Any process at the LHC that produces an $e^+e^-$ pair via a virtual
photon can also produce a $\zeta^+\zeta^-$ pair (we use $\zeta$ as the symbol for an mCP).
This includes the direct vector meson decays $V\to e^+e^-$, where 
$V=\rho,\;\omega,\;\phi,\;\psi,\;\text{or}\;\Upsilon$,
the Dalitz decays $A\to e^+e^-\gamma$, where
$A=\pi^0,\;\eta,\;\text{or}\;\eta'$, and the Dalitz decays 
$\omega\to e^+e^-\pi^0$ and $\eta'\to e^+e^-\omega$.
Additionally, $\zeta^+\zeta^-$ pairs can be produced via
the Drell-Yan process, as with couplings to the photon and
$Z$ of $\varepsilon e$ and $\varepsilon e\tan\theta_w$, respectively.

We generate Drell-Yan decays with \textsc{Madgraph}~\cite{madgraph}, using the
Lagrangian~\ref{eq:mcp_lagr}, with a cut on the invariant mass of the $\zeta^+\zeta^-$
pair of 2\GeV. The Drell-Yan production mode is subdominant when the mass of the mCP
is below half the $\Upsilon$ mass, around 5\GeV.

For mCP pairs produced through meson decay, we perform the two-body/Dalitz decays
manually and store the resulting mCP four-vectors. This requires two pieces of
information for each process: (1) the branching ratio of the meson
to a $\zeta^+\zeta^-$ pair (for Dalitz decays, we need the \textit{differential}
width as a function of the $\zeta^+\zeta^-$ invariant mass), and (2)
the differential cross sections to produce the parent meson as a function of \pt and $\eta$..

Branching ratios for direct vector meson decays can be computed by simply
scaling the $e^+e^-$ BR by a phase space factor:
\be
\frac{\Gamma(V\to\zeta^+\zeta^-)}{\Gamma(V\to e^+e^-)} = 
(Q/e)^2\frac{(1-4x_\zeta^2)^{1/2}(1+2x_\zeta^2)}{(1-4x_e^2)^{1/2}(1+2x_e^2)},
\ee
where $x_*=m_*/m_V$ (this comes from the Van Royen-Wesskpf formula~\cite{vanroyen}).

For Dalitz decays $A\to\zeta^+\zeta^-X$, we can write the differential width as a function of the $\zeta^+\zeta^-$
invariant mass as~\cite{landsberg}
\be
\begin{split}
\frac{d\Gamma}{dq^2} = &\frac{C\alpha}{3\pi q^2} \left(1+\frac{2m_\zeta^2}{q^2}\right)
\sqrt{1-\frac{4m_\zeta^2}{q^2}} \\
&\left[\left(1+\frac{q^2}{m_A^2-m_X^2}\right)-\frac{4m_A^2q^2}{(m_A^2-m_X^2)^2}\right]^{3/2}
\;|F(q^2)|^2 \;\;\Gamma(A\to X\gamma),
\end{split}
\ee
where $q^2$ is the invariant mass of the $\zeta^+\zeta^-$ pair, $C$ is 2 if
$X$ is a $\gamma$ otherwise 1, and $F(q^2)$ is a form factor that can be approximated
in the Vector Dominance Model as
\be
|F(q^2)|^2 = \frac{m_\rho^4+m_\rho^2\Gamma_\rho^2}{(m_\rho^2-q^2)^2+m_\rho^2\Gamma_\rho^2},
\ee
where $m_\rho$ and $\Gamma_\rho$ are the mass and total width of the $\rho$ meson.

Cross sections for the production of parent mesons are acquired in a variety of ways.
For direct~\cite{yqma:jpsi} and 
$B$ meson-mediated~\cite{fonll} production of J/$\psi$
and $\psi'$ mesons, cross sections and \pt distributions (including uncertainties)
are taken directly from theory calculations. Theoretical calculations of $\Upsilon$
production are not reliable at low \pt, so we use differential cross sections
measured by experiment. For $\pt>20\GeV$, we use cross sections measured
at $\sqrt{s}=13\TeV$~\cite{Sirunyan:2017qdw}, and at lower \pt we use measurements
from 7\TeV~\cite{Aad:2012dlq}, rescaled using the measured ration of 13 to 7\TeV cross sections
at slightly higher rapidity.

Differential cross sections for all light-flavor mesons except $\phi$ mesons
are computed by generating minimum bias events in \textsc{pythia8}~\cite{pythia},
with the Monash 2013 tune~\cite{Skands:2014pea}. This is the tune that gives
best agreement with several measurements of light meson rates and \pt spectra
at the LHC~\cite{ALICE-PUBLIC-2018-004,Acharya:2018qnp,Acharya:2017tlv,Sirunyan:2017zmn}, 
albeit in most cases at a center of mass energies lower than 13\TeV.
The MC spectra for $\eta\;(\rho,\omega)$ with $\pt<3\;(1)\GeV$ are scaled
down by factors as large as two, based on these experimental comparisons.
$\phi$ production is modeled with the \textsc{pythia6} generator~\cite{pythia6}
using the DW tune~\cite{Albrow:2006rt}.

\section{Simulation validation and data comparisons}

\section{Background estimation and results}

}
