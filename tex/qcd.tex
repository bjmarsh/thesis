\chapter{QCD Background: The Rebalance and Smear Method}

The third and final background of the \mttwo analysis arises from mis-measured
jets in QCD multijet events. This background is greatly suppressed by the
\mttwo and \dpmin cuts and hence is the smallest of the three backgrounds. 
However, it is also the most difficult to model and estimate since it depends
strongly on the peculiarities of the CMS detector and its imperfect response to jets.
This iteration of the analysis employs a new ``Rebalance and Smear'' method 
to estimate this background. We briefly describe the old method and reasons 
for switching, then explain in detail the new technique.

\section{The $\Delta\phi$-ratio method}
Previous iterations of this analysis \cite{CMS:mt22016,CMS:mt22015} 
used the ``$\Delta\phi$ -ratio'' method to estimate QCD background.

\section{Overview of Rebalance and Smear}

\section{Derivation of jet response templates}

\section{Performance in Monte Carlo}

\section{Electroweak contamination}

\section{Performance in data control regions}

\section{Extension to monojet regions}

\section{Systematic uncertainties}
